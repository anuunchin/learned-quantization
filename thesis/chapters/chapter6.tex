\chapter{Conclusions}
\label{cha:chapter6}

\hspace*{1em}In this thesis, we propose two methods that revolve around 
making ML models learn scale factors to achieve quantization of adjustable intensity. \
In the first approach, 
we introduce a nested quantization layer — one that updates its scale factors through a threshold-based gradient logic. \
Only parameters deemed \textit{non-sensitive} contribute to the decision of whether coarser quantization should be pursued. \
In the second approach, scale factor updates come from custom loss terms —  
MaxBin, Inverse, and Difference — 
each employing distinct logic to guide quantization intensity through a penalty rate. \

Our experiments demonstrate that the nested quantization layer approach
achieves compression rates of up to \( 8 \times \) on the MNIST dataset, and around
\( 6 \times \) on CIFAR-10 and Imagenette, with minimal accuracy degradation.
The optimal penalty threshold was  \( \lambda = 1e-10 \) 
for the MNIST model (where dense layers were quantized),
and \( \lambda = 1e-11 \) 
for CIFAR-10 and Imagenette (where convolutional layers were quantized).
Interestingly, for the MNIST dataset, the added noise from quantization appeared to improve model confidence, 
as evidenced by a lower validation loss relative to the baseline.


Regarding the custom loss terms, they yield compression rates of up to \( 10 \times \) or more.
On MNIST and CIFAR-10, the Difference penalty provided the best results — using row-wise granularity in MNIST and channel-wise granularity in CIFAR-10.
By contrast, on Imagenette, all custom loss terms behaved similarly due to the learning-rate configuration for that model. 
Notably, our post-training quantization-aware training  —  as we put it — resulted in a significant accuracy improvement for Imagenette, 
while yielding performance comparable to the from-scratch quantization scenarios on MNIST and CIFAR-10.

There remains significant room for further exploration, \
especially in extending experiments 
to scenarios where both dense and convolutional layers are quantized simultaneously. \
This could involve employing different penalty thresholds or penalty rates for individual layers. \
Another valid direction would be testing the proposed methods on established architectures such as AlexNet. \
While the nested quantization layer takes parameter sensitivity into account, \
introducing adjustments based on layer-wide sensitivity would be logical — as widely argued in the literature, \
not all layers respond equally to quantization. \
For instance, first and last layers are often left in full precision. \

We acknowlegde that the custom methods introduced in this work add complexity, \
as they require additional computations. 
Moreover, in real-world scenarios, tracking the progression of quantization is difficult for large models — a disadvantage that cannot be ignored when adjusting the intensity of the penalty threshold or rate. \

Given the limits of the current methods, 
future work should focus on improving these methods further while focusing on the theoretical aspect, 
instead of largely relying on experiments. 
Such a shift in focus means clearly defining the optimization problem and proving it can work in theory
by providing a rigorous algorithmic solution.

Nevertheless, our proposed methods are, if not the first, then among the earliest to employ a nested logic — \
a quantization layer that is called within a standard layer. \
The threshold-based gradient method also presents itself as a rather novel apporach, 
taking an alternative direction by focusing on gradient manipulation directly.