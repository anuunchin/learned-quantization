\thispagestyle{empty}
\vspace*{0.2cm}

\begin{center}
    \textbf{Zusammenfassung} \label{zusammenfassung}
\end{center}

\vspace*{0.2cm}

\noindent 
Machine Learning (ML) Modelle sind dafür bekannt,
äußerst ressourcenintensiv zu sein.
Da sie immer häufiger auf Endgeräten zum Einsatz kommen,
wird die Notwendigkeit, Speicher- und Rechenaufwand zu verringern, zunehmend dringlich.
Interessanterweise liefert die Redundanz,
die maßgeblich zur Ineffizienz von neuronalen Netzen beiträgt,
zugleich auch den Ansatzpunkt, um genau diese Ineffizienz zu verringern.
Eine Möglichkeit, die Redundanz von neuronalen Netzen auszunutzen,
ist die Quantisierung –
also die Ersetzung ihrer Präzisionsdarstellungen durch Varianten mit geringerer Bit-Breite.
Unter den verschiedenen Ansätzen der Quantisierung erweist sich
\textit{Learned Quantization}, oder auch erlernte Quantisierung, 
aufgrund des breiten Spektrums an Möglichkeiten,
bei denen ML Modelle selbstständig optimale Quantisierungsstrategien erlernen können,
als ein vielversprechendes Forschungsfeld mit Raum für neue Beiträge.
In dieser Arbeit werden daher zwei Techniken der Learned Quantization vorgestellt,
die eine effektive Modellkompression ermöglichen,
ohne zu wesentlichen Einbußen bei der Genauigkeit zu führen. 
Die erste Technik verwendet speziell angepasste Regularisierungsterme, 
die die Ziele der Quantisierung direkt berücksichtigen. 
Der zweite, neuartige Ansatz integriert eine einzigartige Berechnung des 
Gradienten für den Skalierungsfaktor, 
der den Gradienten der zu quantisierenden Parameter nutzt.
Mit diesen Methoden lässt sich der Speicherbedarf für die Modellparameter
– trainiert auf MNIST, CIFAR-10 und Imagenette
(einer vereinfachten Version von ImageNet)
– um das bis zu Zehnfache reduzieren,
ohne dass die Genauigkeit maßgeblich beeinträchtigt wird.