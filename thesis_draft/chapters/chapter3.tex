\chapter{Learned Quantization\label{cha:chapter3} Schemes}
This chapter introduces two custom learned quantization schemes — approaches that allow models to learn to quantize themselves
with adjustable aggressiveness. The first one, a custom quantization layer featuring tailored logic and a threshold for scale updates,
will be discussed in the first section. The second scheme, which focuses on custom regularization terms with a configurable penalty rate,
will be covered second.

% ------------------------------------------------------------
% ----------------------- learnedquantization Schemes ----------------------- 
% ------------------------------------------------------------

\section{Nested Quantization Layer}
\label{sec:customlayer}
To separate the quantization logic from the usual structure of NN layers,
we define a nested quantization layer that can be used within a standard layer. 
This approach provides usability, making it easy to extend the logic to other types of layers beyond dense and convolutional ones.
The implementation details will follow after we first explain the core logic it incorporates.

% -------------------- tradeoffs and challenges --------------------

\subsection{Concept and Design}
\label{subsec:quantizeddense}
Our nested quantization layer has only one trainable parameter, \textit{scale} (\( s \)),
which serves as the scaling factor for model parameter quantization.
The quantization itself is performed using a simple flooring operation:

\[
  P_{quantized} = floor(\frac{P}{s})
\]

\noindent where \( P \) denotes the parameter being qua`ntized, such as a weight, bias, or kernel.
The scaling factor \( s \) has an adjustable shape, allowing it to be applied at different levels of granularity,
such as per-row, per-column, per-channel, or even per-element. 
\\
\\
During back-propagation, the scaling factor \( s \) is updated using a custom gradient formula. 
The gradient of the loss with respect to \( s \), denoted as \( \nabla_s L \), is computed as:
\[
\nabla_s L = g_s \cdot m,
\]
Let's consider both multiplication terms separately. \(  g_s  \) is the main "decision maker" on whether to
increase scale and therefore quantize more. It is based on a hyperparameter threshold  \(  \lambda  \)
that is compared against against the ratio \(  r  \).

\[
g_s = 
\begin{cases} 
0, & \text{if } r \geq \lambda, \\
-\tanh(r - \lambda), & \text{if } r < \lambda,
\end{cases}
\]
In its turn, \(  r  \) is the raito between the gradient of the model parameter with respect to loss and it's absolute value.
In essence, it conveys how much the parameter 



% -------------------- tradeoffs and challenges --------------------

\subsection{Implementation Details}


\label{subsec:quantizedconvolutional}
In this section, we introduce custom layers built upon Tensorflow's \texttt{tf.keras.Layer} class, 
which serves as the base for all Keras layers. Each custom layer also leverages Tensorflow's 
\texttt{tf.custom\_gradient} decorator to define its own gradient computation.
For clarity, the upcoming subsections start by showing how to define the corresponding standard, 
non-quantized layer using \texttt{tf.keras.Layer} and \texttt{tf.customg\_gradient},
then move on to the specific quantized implementations.

Yang You, Igor Gitman, and Boris Ginsburg - Large batch training of convolutional networks
This might have info on the ratio thingie

\section{Custom Loss Functions}
\label{sec:customloss}

\subsection{Penalty for Inverse Scale Factor Magnitude} 
\label{subsec:scaleinverse}

\subsection{Constraint on Bin Count for Quantization}
\label{subsec:maxbin}

\subsection{Deviation between Quantized and Original Values}
\label{subsec:difference}