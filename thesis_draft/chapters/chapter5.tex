\chapter{Related Work\label{cha:chapter5}}

A significant amount of scientific work has been done on QAT. 
This research can be categorized based on different characteristics, 
which are covered separately in the following paragraphs.

\noindent
\textbf{Model architecture.}\\
RNN - \cite{ott2016rnn}\\
CNN - \cite{rastegari2016xnor}\\
CNN - \cite{courbariaux2015binaryconnect}
DNN - \cite{yunchao2014compressing}
Tranformer bases - \cite{kim2021ibert}

\noindent
\textbf{Quantization target parameters.}\\
weights and activations - \cite{krishnamoorthi2018quantizing}\\
weights and activations - \cite{hubara2016qnn}\\
weights - \cite{polino2018modelcompression}\\
weigths - \cite{ott2016rnn}\\
binary weights and input activations - \cite{rastegari2016xnor}\\
gradients - \cite{shuchang2016dorafenet}
layer inputs and weights - \cite{Edouard2022SPIQ} 
weights an activations -  \cite{DBLP:conf/eccv/ZhangYYH18}
weights activations and gradients - \cite{shuchang2016dorafenet}
\noindent
\textbf{Granularity of quantization.}

\noindent
\textbf{Handling of differentiability.}
STE -  \cite{DBLP:conf/eccv/ZhangYYH18}

\noindent
\textbf{Quantization precision.}\\
binary weights and input activations - \cite{courbariaux2015binaryconnect}\\
binary weights and activations - \cite{hubara2016qnn}\\
binary weights and input activations - \cite{rastegari2016xnor}\\
ternary weights - \cite{ott2016rnn}\\
higher precision for more important parameters and lowe precision for less important ones - \cite{soroosh2018adaptive}\\
mixed precision - \cite{DBLP:conf/eccv/WangLGAC22}
mixed precision - \cite{DBLP:journals/ijcv/DongNLCSZ19}

\noindent
\textbf{Integration with pruning \& other techniques.}\\
pruning and Huffman Codign - \cite{han2016deepcompression}\\
distillation - \cite{polino2018modelcompression}\\
higher precision for more important parameters and lowe precision for less important ones or pruning - \cite{soroosh2018adaptive}
knowledge distillation \cite{DBLP:conf/eccv/WeiPQOY18}

\noindent
\textbf{Modifications to loss functions.}
regularization term WaveQ - \cite{DBLP:journals/corr/abs-2003-00146}\\
proposes a regularization term into the loss function to push the weight values towards +1 and -1 \cite{DBLP:conf/aaai/TangH017}\\
introduces a novel loss formulation where each quantization has different importance - \cite{DBLP:conf/iclr/HouYK17}
\noindent
\textbf{Other interesting approaches.}
k-means for parameters \cite{yunchao2014compressing}

I BERT article has a good related work overview